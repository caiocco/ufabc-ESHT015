% Trabalho de Oficina de Política Urbana
%
% Abaixo seguem orientações originais do modelo utilizado
%
% ==============================================================
%
% Modelo para monografia de final de curso, em conformidade
% com normas da ABNT implementadas pelo projeto abntex2.
%
% Este arquivo é fortemente baseado em exemplo distribuído no
% mesmo projeto. O projeto abntex2 pode ser acessado pela página
% http://abntex2.googlecode.com/
%
% Este arquivo pode ser rodado tanto com o pdflatex quanto com
% o lualatex.  Como contém referências bibliográficas a serem
% processadas pelo programa bibtex, este programa deve ser
% executado. Em resumo, a ordem de execução deve ser:
% rodar primeiro o pdflatex (ou o lualatex), depois o bibtex e,
% a seguir, o pdflatex (ou o lualatex ) novamente mais duas vezes,
% para assegurar que todas as referências bibliográficas e 
% citações estejam atualizadas.
%
% Para adaptar os textos para uso pessoal, usar os comandos
% imediatamente antes do \begin{document} (iniciando com o
% comando \titulo).  
%
% Este modelo está adaptado para monografias de final de curso
% em matemática da UFRJ, mas, com o uso das variáveis, pode ser
% usado para outros tipos de trabalho (mestrado, doutorado),
% outros cursos, universidades etc.  Caso a adaptação das
% variáveis não seja suficiente, pode-se alterar os comandos
% imprimircapa, imprimirfolhaderosto e imprimiraprovação, 
% fazendo as alterações necessárias.  Como os comandos definidos
% neste texto usam somente LaTeX, a sua adaptação deve ser 
% simples, bastando algum conhecimento de LaTeX.
%
% O restante do preâmbulo provavelmente  não necessitará ser
% alterado, a menos, eventualmente, das opções de chamada da
% classe abntex2, que estão definidas a seguir.
% 
\documentclass[ 
% -- opções da classe memoir que é a classe base da abntex2 --
% tamanho da fonte
12pt,
% capítulos começam em pág ímpar. Insere pág vazia, se preciso
openright,
% para imprimir uma página por folha ou visualização em video 
oneside,
% frente e verso. Margens das pag. ímpares diferem das pares.
%  twoside,
% tamanho do papel. 
a4paper,
% Caio - Ocultando bordas horríveis em hiperligações
hidelinks,
% -- opções da classe abntex2 --
% títulos de capítulos convertidos em letras maiúsculas
%  chapter=TITLE,
% títulos de seções convertidos em letras maiúsculas
%  section=TITLE,
% títulos de subseções convertidos em letras maiúsculas
%  subsection=TITLE,
% títulos de subsubseções convertidos em letras maiúsculas
%  subsubsection=TITLE,
% -- opções do pacote babel --
english,   % idioma adicional para hifenização
portuguese,   % o último idioma é o principal do documento
oldfontcommands,
]{abntex2}
%
% ==============================================================
%
% --------------------------------------------------------------
% Adicionando pacotes para recursos adicionais e defindo opções
% pertinentes
% --------------------------------------------------------------
%
% cabeçalho comum para uso com lualatex ou pdflatex
\usepackage{ifluatex}
% opções para uso com o lualatex
\ifluatex
\usepackage{fontspec}
\defaultfontfeatures{Ligatures=TeX}
% o fonte small caps é diferente no latin modern
\fontspec[SmallCapsFont={Latin Modern Roman Caps}]{Latin Modern Roman}
% pacotes da AMS 
\usepackage{amsmath,amsthm} 
% pacote para fonte específico para símbolos matemáticos
\usepackage{unicode-math}
\setmathfont{Latin Modern Math}
% latin modern tem simbolos de mathbb muito feios.
%  Trocar o fonte para estes simbolos.
\setmathfont[range=\mathbb]{Tex Gyre Pagella Math}
% opções para uso com o pdflatex
\else
\usepackage[utf8x]{inputenc}
\usepackage[T1]{fontenc}
\usepackage{lmodern}
\usepackage{etoolbox}
% pacotes da AMS 
\usepackage{amsmath,amssymb,amsthm} 
% Mapear caracteres especiais no PDF
\usepackage{cmap}
\fi

% pacotes usados tanto pelo lualatex quanto pelo pdflatex
\usepackage{lastpage}    % Usado pela Ficha catalográfica
\usepackage{indentfirst} % Indenta primeiro parágrafo 
\usepackage{color}       % Controle das cores
\usepackage{graphicx}    % Inclusão de gráficos
\usepackage{wrapfig}     % gráficos ao redor do texto
% pacote para ajustar os fontes em cada linha de forma a
% respeitar as margens
\usepackage{microtype}
% permite a gravação de texto em um arquivo indicado a partir
% deste arquivo.  Originalmente foi usado para criar o arquivo
% .bib com conteúdo de exemplo, evitando a edição de um arquivo
% .bib somente para a bibliografia
\usepackage{filecontents}

% Caio - diagramas
% http://www.texample.net/tikz/examples/smart-priority/
%\usepackage{smartdiagram}

% Caio - ladeando imagens
% https://tex.stackexchange.com/questions/57433/cannot-use-caption-under-minipage
\usepackage{caption}

% Caio - preciso de tabelas longas
% http://www.tex.ac.uk/FAQ-figurehere.html
\usepackage{longtable}

% Caio - tentando melhorar o posicionamento das imagens
\usepackage{float}

% Caio - corrigindo espaçamento conforme http://tex.stackexchange.com/questions/5683/how-to-remove-top-and-bottom-whitespace-of-longtable
\setlength{\LTpre}{0pt}
\setlength{\LTpost}{0pt}

% Caio - preciso de plotagens
%\usepackage{pgfplots}
%\pgfplotsset{compat=1.8}

% Caio - quero usar letras nas listas do enumerate conforme https://tex.stackexchange.com/questions/2291/how-do-i-change-the-enumerate-list-format-to-use-letters-instead-of-the-defaul
\usepackage{enumitem}

% Caio - modo paisagem para tabelões
\usepackage{lscape}

% Caio - adicionando o pacote hyperref
\usepackage{hyperref}
% - e definindo metadados do PDF e comportamento dos links
\hypersetup{
	%pagebackref=true,
	pdftitle={Relatório/Diagnóstico de Política Urbana}, 
	pdfauthor={Anderson, Caio, Jade, Henrique, Noele},
	pdfsubject={Política Urbana},
	colorlinks=false,      		% false: boxed links; true: colored links
	linkcolor=blue,          	% color of internal links
	citecolor=blue,        		% color of links to bibliography
	filecolor=magenta,      	% color of file links
	urlcolor=blue,
	bookmarksdepth=4
}

% Caio - separação silábica
%\hyphenation{}

% Caio - citações mais poderosas
%\usepackage[autostyle]{csquotes}

%-----------------------------------------------------------
%-----------------------------------------------------------
% Caio - habilitar glossário
\usepackage{glossaries}
\makeglossaries

% \newglossaryentry{ex}{name={sample},description={an example}}
\newglossaryentry{abl}{
	name={ABL},
	description={Área Bruta Locável}
}

\newglossaryentry{auj}{
	name={AUJ},
	description={Aglomeração Urbana de Jundiaí}
}

\newglossaryentry{condephaat}{
	name={CONDEPHAAT},
	description={Conselho de Defesa do Patrimônio Histórico, Arqueológico, Artístico e Turístico}
}

\newglossaryentry{cptm}{
	name={CPTM},
	description={Companhia Paulista de Trens Metropolitanos}
}

\newglossaryentry{tim}{
	name={TIM},
	description={Trem Intra-Metropolitano}
}

\newglossaryentry{cmsp}{
	name={CMSP},
	description={Companhia do Metropolitano de São Paulo}
}

\newglossaryentry{embraesp}{
	name={Embraesp},
	description={Empresa Brasileira de Estudos de Patrimônio}
}

\newglossaryentry{emtu}{
	name={EMTU},
	description={Empresa Metropolitana de Transportes Urbanos de São Paulo S.A}
}

\newglossaryentry{emplasa}{
	name={Emplasa},
	description={Empresa Paulista de Planejamento Metropolitano S/A}
}

\newglossaryentry{luos}{
	name={LUOS},
	description={Lei de Uso de Ocupação do Solo}
}

\newglossaryentry{mdu}{
	name={MDU},
	description={Média por Dia Útil}
}

\newglossaryentry{ouc}{
	name={OUC},
	description={Operação Urbana Consorciada}
}

\newglossaryentry{pde}{
	name={PDE},
	description={Plano Diretor Estratégico}
}

\newglossaryentry{peuc}{name={PEUC},description={Parcelamento, Edificação e Utilização Compulsórios}}

\newglossaryentry{pl}{
	name={PL},
	description={Projeto de Lei}
}

\newglossaryentry{rmsp}{
	name={RMSP},
	description={Região Metropolitana de São Paulo}
}

\newglossaryentry{rmbs}{
	name={RMBS},
	description={Região Metropolitana da Baixada Santista}
}

\newglossaryentry{cetesb}{
	name={Cetesb},
	description={Companhia Ambiental do Estado de São Paulo}
}

\newglossaryentry{sapavel}{
	name={SAPAVEL},
	description={Sistema de Áreas Protegidas, Áreas Verdes e Espaços Livres}
}

\newglossaryentry{smdu}{
	name={SMDU},
	description={Secretaria Municipal de Desenvolvimento Urbano da Prefeitura de São Paulo}
}

\newglossaryentry{idhm}{
	name={IDHM},
	description={Índice de Desenvolvimento Humano}
}

\newglossaryentry{efs}{
	name={EFS},
	description={Estrada de Ferro Sorocabana}
}

\newglossaryentry{vlt}{
	name={VLT},
	description={Veículo Leve sobre Trilhos}
}

\newglossaryentry{brt}{
	name={BRT},
	description={Bus Rapid Transit}
}

\newglossaryentry{rffsa}{
	name={RFFSA},
	description={Rede Ferroviária Federal Sociedade Anônima}
}

\newglossaryentry{iptu}{
	name={IPTU},
	description={Imposto Predial e Territorial Urbano}
}

\newglossaryentry{fepasa}{
	name={Fepasa},
	description={Ferrovia Paulista Sociedade Anônima}
}

\newglossaryentry{iss}{
	name={ISS},
	description={Imposto Sobre Serviços de Qualquer Natureza}
}

\newglossaryentry{app}{
	name={APP},
	description={Área de Preservação Permanente}
}

\newglossaryentry{luops}{
	name={LUOPS},
	description={Legislação de Ordenamento do Uso, da Ocupação e do Parcelamento do Solo}
}

\newglossaryentry{plhis}{
	name={PLHIS},
	description={Plano Local de Habitação de Interesse Social}
}

\newglossaryentry{sabesp}{
	name={Sabesp},
	description={Companhia de Saneamento Básico do Estado de São Paulo}
}

\newglossaryentry{pddmap}{
	name={PDDMAP},
	description={Plano Diretor de Drenagem e Manejo de Águas Pluviais}
}

\newglossaryentry{agem}{
	name={AGEMBS},
	description={Agência Metropolitana da Baixada Santista}
}

\newglossaryentry{pac}{
	name={PAC},
	description={Programa de Aceleração do Crescimento}
}

\newglossaryentry{zeis}{
	name={ZEIS},
	description={Zona Especial de Interesse Social}
}

\newglossaryentry{ac1}{
	name={AC-1},
	description={Clubes esportivos sociais}
}

\newglossaryentry{ac2}{
	name={AC-2},
	description={Clubes de campo e clubes náuticos}
}

\newglossaryentry{zc}{
	name={ZC},
	description={Zona Centralidade}
}

\newglossaryentry{zczeis}{
	name={ZC-ZEIS},
	description={Zona Centralidade lindeira à ZEIS}
}

\newglossaryentry{zca}{
	name={ZCa},
	description={Zona Centralidade Ambiental}
}

\newglossaryentry{zcor1}{
	name={ZCOR-1},
	description={Zona Corredor 1}
}

\newglossaryentry{zcor2}{
	name={ZCOR-2},
	description={Zona Corredor 2}
}

\newglossaryentry{zcor3}{
	name={ZCOR-3},
	description={Zona Corredor 3}
}

\newglossaryentry{zcora}{
	name={ZCORa},
	description={Zona Corredor Ambiental}
}

\newglossaryentry{zde1}{
	name={ZDE-1},
	description={Zona de Desenvolvimento Econômico 1}
}

\newglossaryentry{zde2}{
	name={ZDE-2},
	description={Zona de Desenvolvimento Econômico 2}
}

\newglossaryentry{zeis1}{
	name={ZEIS-1},
	description={Zona Especial de Interesse Social 1}
}

\newglossaryentry{zeis2}{
	name={ZEIS-2},
	description={Zona Especial de Interesse Social 2}
}

\newglossaryentry{zeis3}{
	name={ZEIS-3},
	description={Zona Especial de Interesse Social 3}
}

\newglossaryentry{zeis4}{
	name={ZEIS-4},
	description={Zona Especial de Interesse Social 4}
}

\newglossaryentry{zeis5}{
	name={ZEIS-5},
	description={Zona Especial de Interesse Social 5}
}

\newglossaryentry{zem}{
	name={ZEM},
	description={Zona Eixo de Estruturação Transformação Metropolitana}
}

\newglossaryentry{zemp}{
	name={ZEMP},
	description={Zona Eixo de Estruturação Transformação Metropolitana Previsto}
}

\newglossaryentry{zep}{
	name={ZEP},
	description={Zona Especial de Preservação}
}

\newglossaryentry{zepam}{
	name={ZEPAM},
	description={Zona Especial de Proteção Ambiental}
}

\newglossaryentry{zer1}{
	name={ZER-1},
	description={Zona Exclusivamente Residencial 1}
}

\newglossaryentry{zer2}{
	name={ZER-2},
	description={Zona Exclusivamente Residencial 2}
}

\newglossaryentry{zera}{
	name={ZERa},
	description={Zona Exclusivamente Residencial Ambiental}
}

\newglossaryentry{zeu}{
	name={ZEU},
	description={Zona Eixo de Estruturação da Transformação Urbana}
}

\newglossaryentry{zeua}{
	name={ZEUa},
	description={Zona Eixo de Estruturação da Transformação Urbana Ambiental}
}

\newglossaryentry{zeup}{
	name={ZEUP},
	description={Zona Eixo de Estruturação da Transformação Previsto}
}

\newglossaryentry{zeupa}{
	name={ZEUPa},
	description={Zona Eixo de Estruturação da Transformação Previsto Ambiental}
}

\newglossaryentry{zm}{
	name={ZM},
	description={Zona Mista}
}

\newglossaryentry{zma}{
	name={ZMa},
	description={Zona Mista Ambiental}
}

\newglossaryentry{zmis}{
	name={ZMIS},
	description={Zona Mista de Interesse Social}
}

\newglossaryentry{zmisa}{
	name={ZMISa},
	description={Zona Mista de Interesse Social Ambiental}
}

\newglossaryentry{zoe}{
	name={ZOE},
	description={Zona de Ocupação Especial}
}

\newglossaryentry{zpds}{
	name={ZPDS},
	description={Zona de Preservação e Desenvolvimento Sustentável}
}

\newglossaryentry{zpdsr}{
	name={ZPDSr},
	description={Zona de Preservação e Desenvolvimento Sustentável da Zona Rural}
}

\newglossaryentry{zpi1}{
	name={ZPI-1},
	description={Zona Predominantemente Industrial 1}
}

\newglossaryentry{zpi2}{
	name={ZPI-2},
	description={Zona Predominantemente Industrial 1}
}

\newglossaryentry{zpr}{
	name={ZPR},
	description={Zona Predominantemente Residencial}
}

\newglossaryentry{pmsp}{
	name={PMSP},
	description={Prefeitura do Município de São Paulo}
}

\newglossaryentry{smpr}{
	name={SMPR},
	description={Secretaria Municipal de Prefeituras Regionais}
}

\newglossaryentry{smg}{
	name={SMG},
	description={Secretaria Municipal de Gestão}
}

\newglossaryentry{ibge}{
	name={IBGE},
	description={Instituto Brasileiro de Geografia e Estatística}
}

\newglossaryentry{gesp}{
	name={GESP},
	description={Governo do Estado de São Paulo}
}

%-----------------------------------------------------------
%-----------------------------------------------------------
% Comandos para definir ambientes tipo teorema em português 
\newtheorem{meuteorema}{Teorema}[chapter]
\newtheorem{meuaxioma}{Axioma}[chapter]
\newtheorem{meucorolario}{Corolário}[chapter]
\newtheorem{meulema}{Lema}[chapter]
\newtheorem{minhaproposicao}{Proposição}[chapter]
\newtheorem{minhadefinicao}{Definição}[chapter]
\newtheorem{meuexemplo}{Exemplo}[chapter]
\newtheorem{minhaobservacao}{Observação}[chapter]
%-----------------------------------------------------------
%-----------------------------------------------------------
% Pacotes de citações
\usepackage[brazilian,hyperpageref]{backref}
\usepackage[alf]{abntex2cite}   % Citações padrão ABNT
%\usepackage[num]{abntex2cite}  % Citações numéricas
% --- 
% Configurações do pacote backref
% Usado sem a opção hyperpageref de backref
\renewcommand{\backrefpagesname}{Citado na(s) página(s):~}
% Texto padrão antes do número das páginas
\renewcommand{\backref}{}
% Define os textos da citação
\renewcommand*{\backrefalt}[4]{
	\ifcase #1 %
	Nenhuma citação no texto.%
	\or
	Citado na página #2.%
	\else
	Citado #1 vezes nas páginas #2.%
	\fi}%
% --- 
% --- 
% Espaço em branco no início do parágrafo
\setlength{\parindent}{1.3cm}
% Controle do espaçamento entre um parágrafo e outro:
\setlength{\parskip}{0.2cm}  % tente também \onelineskip
% ---
% compila o indice, se este for incluído no texto
\makeindex
%
% --------------------------------------------------------- 
% ---------------------------------------------------------
% Redefinindo o comando do abntex2 para gerar uma capa  
\renewcommand{\imprimircapa}{%
	\begin{capa}
	\begin{flushleft} 
		{\Large \textsc{\imprimirinstituicao  \\
				\imprimircurso \\} }
	\end{flushleft}
	
	\vfill
	\begin{center}
		{\large \imprimirautor} \\
		{\Large \textit{\imprimirtitulo}}
	\end{center}
	
	\vfill
	\begin{center}
		{\large{\imprimirlocal \\ \imprimirano  }}
	\end{center}
	\vspace*{1cm} 
	\end{capa}
	
}

% ---------------------------------------------------------
% ---------------------------------------------------------
%
%
% ---------------------------------------------------------
% ---------------------------------------------------------
% Redefinindo o comando para gerar uma folha de rosto 
\renewcommand{\imprimirfolhaderosto}{%
	\begin{center}
		{\large \imprimirautor}
	\end{center}
	\vfill \vfill \vfill \vfill
	\begin{center}
		{\Large \textit{\imprimirtitulo}}
	\end{center}
	
	\vfill \vfill \vfill 
	\begin{flushright} 
		\parbox{0.5\linewidth}{
			\imprimirtipotrabalho\, relacionado ao 
			\imprimircurso\, da \imprimirsigla\, 
			entregue como parte do
			processo de graduação para a obtenção do 
			grau de \imprimirgrau.}
	\end{flushright} 
	
	\vfill 
	\begin{flushright} 
		\parbox{0.5\linewidth}{ \imprimirorientadorRotulo 
			\imprimirorientador\\ \imprimirttorientador}
	\end{flushright} 
	
	\ifdefvoid{\imprimircoorientador}{}{
		\begin{flushright} 
			\parbox{0.5\linewidth}{ \imprimircoorientadorRotulo 
				\imprimircoorientador\\ \imprimirttcoorientador}
		\end{flushright}
	}
	
	\vfill \vfill \vfill \vfill \vfill \vfill \vfill
	\begin{center}
		{\large{\imprimirlocal \\ \imprimirano}}
	\end{center}
	\vspace*{1cm} \newpage
}
% Final do comando para gerar uma folha de rosto 
% ---------------------------------------------------------
% ---------------------------------------------------------
%
%
% ---------------------------------------------------------
% ---------------------------------------------------------
% Definindo o comando para gerar uma folha de defesa 
\newcommand{\imprimirfolhadeaprovacao}{%
	\begin{center}
		{\large \imprimirautor}
	\end{center}
	\vfill \vfill \vfill \vfill
	\begin{center}
		{\Large \textit{\imprimirtitulo}}
	\end{center}
	
	\vfill \vfill \vfill \vfill \vfill \vfill
	\begin{flushright} 
		\parbox{0.5\linewidth}{
%			\imprimirtipotrabalho\,apresentada ao 
%			\imprimircurso\, da \imprimirsigla\, como requisito
%			para a obtenção parcial do grau de \imprimirgrau.}
		}
	\end{flushright} 
	\vfill \vfill \vfill \vfill
	Aprovada em \data.
	
	\vfill \vfill \vfill \vfill
	
	\begin{center}
		\textbf{BANCA EXAMINADORA}
		
		\vfill\vfill\vfill
		\rule{10cm}{.1pt}\\
		{\imprimirexaminadorum} \\ {\imprimirttexaminadorum}
		
		\ifdefvoid{\imprimirexaminadordois}{}{
			\vfill\vfill
			\rule{10cm}{.1pt}\\
			\imprimirexaminadordois \\ \imprimirttexaminadordois }
		
		\ifdefvoid{\imprimirexaminadortres}{}{
			\vfill\vfill
			\rule{10cm}{.1pt}\\
			\imprimirexaminadortres \\ \imprimirttexaminadortres }
		
		\ifdefvoid{\imprimirexaminadorquatro}{}{
			\vfill\vfill
			\rule{10cm}{.1pt}\\
			\imprimirexaminadorquatro \\ \imprimirttexaminadorquatro }
	\end{center}
	
	\vfill \vfill 
	\begin{center}
		{\large{\imprimirlocal \\ \imprimirano}}
	\end{center}
	\vspace*{1cm}
	\newpage
}
% Final do comando para gerar uma folha de defesa 
% ---------------------------------------------------------
% --------------------------------------------------------
%
%
%
%
%
% ---------------------------------------------------------
% --------------------------------------------------------
% definindo variáveis adicionais 
\providecommand{\imprimirsigla}{}
\newcommand{\sigla}[1]{\renewcommand{\imprimirsigla}{#1}}
%
\providecommand{\imprimircurso}{}
\newcommand{\curso}[1]{\renewcommand{\imprimircurso}{#1}}
%
\providecommand{\imprimirano}{}
\newcommand{\ano}[1]{\renewcommand{\imprimirano}{#1}}
%
\providecommand{\imprimirgrau}{}
\newcommand{\grau}[1]{\renewcommand{\imprimirgrau}{#1}}
%
\providecommand{\imprimirexaminadorum}{}
\newcommand{\examinadorum}[1]{
	\renewcommand{\imprimirexaminadorum}{#1}}
%
\providecommand{\imprimirexaminadordois}{}
\newcommand{\examinadordois}[1]{
	\renewcommand{\imprimirexaminadordois}{#1}}
%
\providecommand{\imprimirexaminadortres}{}
\newcommand{\examinadortres}[1]{
	\renewcommand{\imprimirexaminadortres}{#1}}
%
\providecommand{\imprimirexaminadorquatro}{}
\newcommand{\examinadorquatro}[1]{
	\renewcommand{\imprimirexaminadorquatro}{#1}}
%
\providecommand{\imprimirttorientador}{}
\newcommand{\ttorientador}[1]{
	\renewcommand{\imprimirttorientador}{#1}} 
%
\providecommand{\imprimirttcoorientador}{}
\newcommand{\ttcoorientador}[1]{
	\renewcommand{\imprimirttcoorientador}{#1}}
%
\providecommand{\imprimirttexaminadorum}{}
\newcommand{\ttexaminadorum}[1]{
	\renewcommand{\imprimirttexaminadorum}{#1}}
%
\providecommand{\imprimirttexaminadordois}{}
\newcommand{\ttexaminadordois}[1]{\renewcommand{
		\imprimirttexaminadordois}{#1}}
%
\providecommand{\imprimirttexaminadortres}{}
\newcommand{\ttexaminadortres}[1]{
	\renewcommand{\imprimirttexaminadortres}{#1}}
%
\providecommand{\imprimirttexaminadorquatro}{}
\newcommand{\ttexaminadorquatro}[1]{
	\renewcommand{\imprimirttexaminadorquatro}{#1}}
% fim da definição de variáveis adicionais
% ---------------------------------------------------------
% ---------------------------------------------------------
%
% ---
% ---
% ---
% ---
% ---
% ---
% ---
% ---
% ---
% Informações de dados para CAPA, FOLHA DE ROSTO e FOLHA DE DEFESA
%
%----------------- Título e Dados do Autor -----------------
\titulo{Diagnóstico \& Proposta}
\autor{Bruna Fernandes \and
	   Caio César C. Ortega \and
	   Jade Vieira Cavalhieri \and
	   Leonardo \and
	   Luciana Akemi
}
%

%----------Informações sobre a Instituição e curso -----------------
\instituicao{Universidade Federal do ABC \\
	Centro de Engenharia, Modelagem e Ciências Sociais Aplicadas}
%
\sigla{UFABC}
%
\curso{Bacharelado em Planejamento Territorial}
%\curso{Curso de Licenciatura em Matemática}
%\curso{Mestrado em Ensino de Matemática}
%\curso{Doutorado em Matemática}
%
\local{São Bernardo do Campo, SP}
%
%
% -------- Informações sobre o tipo de documento
\tipotrabalho{Relatório}
%\tipotrabalho{Monografia de final de curso}
%\tipotrabalho{Dissertação de mestrado}
%\tipotrabalho{Tese de doutorado}
%
\grau{BACHAREL em Planejamento Territorial}
%\grau{LICENCIADO em Matemática}
%\grau{MESTRE em Matemática}
%\grau{DOUTOR em Ciências}
%
\ano{2018}
\data{13 de Junho de 2018} % data da aprovação
%
%------Nomes do Orientador, examinadores.  
\orientador{Érico R. P. Novais}
%\coorientador{Antonio da Silva} % opcional
\examinadorum{Érico R. P. Novais}
%\examinadordois{Ivo Fernandez Lopez}
%\examinadortres{Jeferson Leandro Garcia de Araújo}
%\examinadorquatro{Antonio da Silva}
%
%--------- Títulos do Orientador e examinadores ----
%\ttorientador{Bacharel em Física - UEFS}
%\ttcoorientador{Doutor em Matemática - UFRJ} 
%\ttexaminadorum{Doutor em Matemática - UFRJ}
%\ttexaminadordois{Doutor em Matemática - UFRJ}
%\ttexaminadortres{Doutor em Matemática - UFRJ}
%\ttexaminadorquatro{Doutor em Matemática - UFRJ}
%
% ---
% ---
\begin{document}
		
	% ---
	% Chamando o comando para imprimir a capa
	\imprimircapa
	% ---
	% ---
	% Chamando o comando para imprimir a folha de rosto
	%\imprimirfolhaderosto
	% ---
	% ---
	% Chamando o comando para imprimir a folha de aprovação
	%\imprimirfolhadeaprovacao
	% ---
	% ---
	% Dedicatória
	% ---
	%	\begin{dedicatoria}
	%  	 \vspace*{\fill}
	%  	 \centering
	%  	 \noindent
	%  	 \textit{ Este trabalho é dedicado a todos que, com entusiasmo,\\
	%  	 		sonham e lutam por XYZ no ABCDEFG\\
	%  			do XPTO.} \vspace*{\fill}
	%	\end{dedicatoria}
	%	
	%	
	%	\begin{agradecimentos}
	%	Orientação do modelo: insira aqui um parágrafo
	%	\end{agradecimentos}
	%	
	%	
	%
	%---------------------- EPÍGRAFE I (OPCIONAL)--------------
	%\begin{epigrafe}
	%    \vspace*{\fill}
	%    \begin{flushright}
	%        \textit{''Texto''\\
	%        Autor}
	%    \end{flushright}
	%\end{epigrafe}
	%
	%
	%
	%--------Digite aqui o seu resumo em %Português--------------
	%\begin{resumo}
	%   Descrição. 
	%
	%   \vspace{\onelineskip}
	%   \noindent
	%   \textbf{Palavras-chaves}: Palavras.
	%\end{resumo}
	
	
	%
	% --- resumo em inglês (abstract) ---
	%\begin{resumo}[Abstract]
	%   \begin{otherlanguage*}{english}
	%      Description.
	%
	%      \vspace{\onelineskip}
	%      \noindent
	%      \textbf{Keywords}: Words.
	%   \end{otherlanguage*}
	%\end{resumo}

	%
	%----Sumário, lista de figura e de tabela ------------
	\tableofcontents 
	\newpage \listoffigures
	\newpage \listoftables
	%---------------------
	%--------------Início do Conteúdo---------------------------
	% o comando textual é obrigatório e marca o ponto onde começa 
	% a imprimir o número da página
	\textual
	%
	%---------------------
	%


%
% O conteúdo começa pra valer a seguir
%

%
%===============================================================================
%
	
	\chapter{Introdução}

	Lorem ipsum\dots	
    
%
%===============================================================================
%

	% ----------------------------------------------------------
	% ----------------------------------------------------------
	\postextual
	
	
	
	% informa o arquivo com a bibliografia. Deve ser o mesmo nome
	% (sem o sufixo) que será informado no ambiente filecontents
	% que está no final deste arquivo. Neste exemplo foi usado 
	% bibitemp.bib e bibtemp. Este comando insere a bibliografia
	% nesta posição (antes dos apêndices, anexos, índice remissivo)
	\bibliography{fontes}
	% ----------------------------------------------------------
	% Glossário
	% ----------------------------------------------------------
	% Consultar manual da classe abntex2 para orientações sobre o
	% uso do glossário.
	\renewcommand{\glossaryname}{Glossário}
	%\renewcommand{\glossarypreamble}{Esta é a descrição do glossário.\\ \\}
	\renewcommand*{\glsseeformat}[3][\seename]{\textit{#1}
		\glsseelist{#2}}
	
	% ---
	% Traduções para o ambiente glossaries
	% ---
	\providetranslation{Glossary}{Glossário}
	\providetranslation{Acronyms}{Siglas}
	\providetranslation{Notation (glossaries)}{Notação}
	\providetranslation{Description (glossaries)}{Descrição}
	\providetranslation{Symbol (glossaries)}{Símbolo}
	\providetranslation{Page List (glossaries)}{Lista de Páginas}
	\providetranslation{Symbols (glossaries)}{Símbolos}
	\providetranslation{Numbers (glossaries)}{Números} 
	% ---
	
	% ---
	% Imprime o glossário
	% ---
	\cleardoublepage
	\phantomsection
	\addcontentsline{toc}{chapter}{\glossaryname}
	% \glossarystyle{index}
	% \glossarystyle{altlisthypergroup}
	\glossarystyle{tree}
	\printglossaries
	% ---
	
	% ----------------------------------------------------------
	% Apêndices
	% ----------------------------------------------------------
	
	% ---
	% Inicia os apêndices. Não esquecer de fechar ao final de
	% todos os apêndices (\end{apendicesenv})
	% ---
	%\begin{apendicesenv}
	
	% Imprime uma página indicando o início dos apêndices
	%\partapendices
	
	% ----------------------------------------------------------
	%\chapter{Primeiro apêndice}
	% ----------------------------------------------------------
	
	%Este é um exemplo de inclusão de capítulos de %apêndice em uma 
	%monografia.  Cada apêndice é tratado como se fosse %um capítulo.
	%Os apêndices devem ser iniciados pelo comando de %ambiente
	%\textbackslash begin\{apendicesenv\} e encerrados %pelo comando 
	%\textbackslash end\{apendicesenv\}.
	
	% ----------------------------------------------------------
	%\chapter{Segundo apêndice}
	% ----------------------------------------------------------
	
	%Este é um exemplo de inclusão de um segundo apêndice. 
	
	%\end{apendicesenv}
	% ---
	
	
	% ----------------------------------------------------------
	% Anexos
	% ----------------------------------------------------------
	
	% ---
	% Inicia os anexos
	% ---
	%\begin{anexosenv}
	
	% Imprime uma página indicando o início dos anexos
	%\partanexos
	
	% ---
	%\chapter{Primeiro anexo}
	% ---
	%Os anexos são similares aos apêndices se distinguindo pelo fato
	%que os apêndices são de autoria do autor da monografia e os 
	%anexos não são da autoria do autor da monografia.  Por exemplo,
	%se incluir no trabalho um modelo de um formulário preenchido
	%por alunos participantes de uma pesquisa, este será um apêndice
	%se o formulário foi criado pelo autor da monografia e será um
	%anexo se o formulário tiver sido criado por outros (por exemplo,
	%é um formulário padrão da escola em que o aluno que o preenche
	%estuda).
	%
	%Mesmo que o formulário tenha sido elaborado pela escola, uma
	%reprodução do formulário preenchido por cada aluno na pesquisa
	%será incluído no apêndice pois envolve o trabalho do autor da
	%monografia ao distribuir, coletar e reproduzir as respostas.
	%
	%Este é um exemplo de inclusão de capítulos de anexos em uma 
	%monografia.  Cada anexo é tratado como se fosse um capítulo.
	%Os anexos devem ser iniciados pelo comando de ambiente
	%\textbackslash begin\{anexoenv\} e encerrados pelo comando 
	%\textbackslash end\{anexoenv\}.
	%
	%\end{anexosenv}
	% ---
	%---------------------------------------------------------------------
	%---------------------------------------------------------------------
	
	%\printindex
	
	% Por padrão são incluídas no trabalho somente as referências
	% citadas ao longo do texto. No comando abaixo foram acrescentadas
	% algumas referências não citadas (neste texto servem apenas como
	% exemplos). Não deve ser usado o comando (mais simples) 
	% \nocite{*}, pois este parece não ser compatível com o
	% abntex2cite
	%\nocite{abntex2cite,abntex2wiki,boyer,eves,iezzi,kletenic,
	%        diomara,steinbruch,intusolatex,feynman,shannon,
	%        luisfelipe,turing}
\end{document}
